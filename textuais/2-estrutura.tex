\section{Estrutura do artigo}

A estrutura do artigo é constituída de elementos pré-textuais, textuais e pós-textuais. Conforme esquema descrito na Tabela~\ref{tbl1}.

\begin{table}[h]
	\centering
	\label{tbl1}
	\caption{Estrutura dos artigos}
	\begin{tabularx}{\textwidth}{cX} \toprule
		Elementos & Seções \\ \midrule
		\multirow{6}{*}{Pré-textuais} & \textbf{Título, e subtítulo (se houver) (obrigatório)} \\
		& \textbf{Nome(s) do(s) autor(es) (obrigatório)} \\ 
		& \textbf{Resumo na língua do texto (obrigatório)} \\
		& Resumo em outro idioma (opcional) \\
		& \textbf{Datas de submissão e aprovação do artigo (obrigatório)} \\
		& Identificação e disponibilidade (opcional) \\ \midrule
		\multirow{3}{*}{Textuais} & \textbf{Introdução (obrigatório)} \\ 
		& \textbf{Desenvolvimento (obrigatório)} \\ 
		& \textbf{Considerações finais (obrigatório)} \\ \midrule
		\multirow{5}{*}{Pós-textuais} & \textbf{Referências (obrigatório)} \\
		& Glossário (opcional) \\
		& Apêndice(s) (opcional) \\
		& Anexo(s) (opcional) \\
		& Agradecimentos (opcional)  \\ \bottomrule
	\end{tabularx} 
	\fonte{Baseado em \citeonline{guiaTCC}.}
\end{table}