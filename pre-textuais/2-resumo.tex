% página de título principal (obrigatório)
\maketitle

% título em outro idioma (opcional)
\noindent\rule{\textwidth}{1pt}

% resumo em português
\begin{resumoumacoluna}
    Conforme a ABNT NBR 6022:2018, o resumo no idioma do documento é elemento obrigatório. Constituído de uma sequência de frases concisas e objetivas e não de uma simples enumeração de tópicos, não ultrapassando 250 palavras, seguido, logo abaixo, das palavras representativas do conteúdo do trabalho, isto é, palavras-chave e/ou descritores, conforme a NBR 6028. (\ldots) As palavras-chave devem figurar logo abaixo do resumo, antecedidas da expressão Palavras-chave: entre 3 e 5, separadas entre si por ponto e finalizadas também por ponto.
   
    \vspace{\onelineskip}
    
    \noindent
    \textbf{Palavras-chave}: latex. abntex. editoração de texto.
\end{resumoumacoluna}


% resumo em língua estrangeira, opcional
\renewcommand{\resumoname}{Abstract}
\begin{resumoumacoluna}
\begin{otherlanguage*}{english}
    According to ABNT NBR 6022:2018, an abstract in foreign language is optional.

    \vspace{\onelineskip}

    \noindent
    \textbf{Keywords}: latex. abntex.
\end{otherlanguage*}  
\end{resumoumacoluna}


\begin{center}\smaller
    \textbf{Data de submissão}: elemento obrigatório. Indicar dia, mês e ano \\
    \textbf{Data de aprovação}: elemento obrigatório. Indicar dia, mês e ano
    % \textbf{Identificação e disponibilidade}: elemento opcional. Pode ser indicado 
    % o endereço eletrônico, DOI, suportes e outras informações relativas ao acesso.
\end{center}